\documentclass[12pt,a4paper]{article}
\usepackage{amsfonts, amssymb, amsmath}
\usepackage{fullpage}
\usepackage{parskip} % skip a line instead of indenting
\usepackage{amsthm}
\usepackage{xcolor}

\newtheorem*{rem}{Remark}

\title{Solving Linear Equations}
\author{R4 Cheng}
\date{\today}

\newcommand{\Remark}[1]{
  \begin{rem}
    \color{cyan}
    #1
  \end{rem}
}

\begin{document}
\maketitle

\subsection*{Matrix Operations}

$
\begin{bmatrix}
  1 & 2 & -4 \\
  -2 & 3 & 1 \\
  4 & 1 & 2
\end{bmatrix}
\begin{bmatrix}
  x_1 & x_2 & x_3 \\
  y_1 & y_2 & y_3 \\
  z_1 & z_2 & z_3 
\end{bmatrix} = 
\begin{bmatrix}
  & \\
  \underline{c_1} & \underline{c_2} & \underline{c_3} \\
  &
\end{bmatrix}
$

$
\underline{c_2} = 
x_2
\begin{bmatrix}
  1 \\
  -2 \\
  4
\end{bmatrix} + 
y_2
\begin{bmatrix}
  2 \\
  3 \\
  1
\end{bmatrix} +
z_2
\begin{bmatrix}
  -4 \\
  1 \\
  2
\end{bmatrix}
$

$
\begin{bmatrix}
  a & b & c \\
\end{bmatrix}
\begin{bmatrix}
  1 & 2 & -4 \\
  -2 & 3 & 1 \\
  4 & 1 & 2
\end{bmatrix} = 
a
\begin{bmatrix}
  1 & 2 & -4 \\
\end{bmatrix} +
b
\begin{bmatrix}
  -2 & 3 & 1 \\
\end{bmatrix} +
c
\begin{bmatrix}
  4 & 1 & 2
\end{bmatrix}
$


\subsection*{Properties of Matrices}

$A(BC) = (AB)C$ (Associative law holds)

$AB \neq BA$ (Commutative law does not hold)

$C(A+B) = CA + CB$ or $(A+B)C = AC + BC$ (Distributive laws hold)

\Remark{We can change Guassian Elimination to Matrix multiplication}

\subsection*{Identity Matrix}
The identity matrix is a square matrix with ones on the main diagonal and zeros elsewhere. 
It is denoted by $I$ or $I_n$ for an $n \times n$ matrix.
$AI = IA = A$, for any $n \times n$ matrix $A$.

\[
I = \begin{bmatrix}
1 & 0 & \cdots & 0 \\
0 & 1 & \cdots & 0 \\
\vdots & \vdots & \ddots & \vdots \\
0 & 0 & \cdots & 1
\end{bmatrix}
\]


\subsection*{Inverse Matrix}
The inverse of a square matrix $A$, denoted as $A^{-1}$, is a matrix such that $AA^{-1} = A^{-1}A = I$, where $I$ is the identity matrix. The inverse matrix can be found using the formula:

\[
A^{-1} = \frac{1}{{\text{det}(A)}} \cdot \text{adj}(A)
\]

where $\text{det}(A)$ is the determinant of matrix $A$ and $\text{adj}(A)$ is the adjugate of matrix $A$.

For example, to find the inverse of a $2 \times 2$ matrix:

\[
A = \begin{bmatrix}
a & b \\
c & d
\end{bmatrix}
\]

the inverse matrix is given by:

\[
A^{-1} = \frac{1}{{ad - bc}} \cdot \begin{bmatrix}
d & -b \\
-c & a
\end{bmatrix}
\]

To find the inverse of a $3 \times 3$ matrix, you can use the formula:

\[
A^{-1} = \frac{1}{{\text{det}(A)}} \cdot \text{adj}(A)
\]

where $\text{adj}(A)$ is the adjugate of matrix $A$. The adjugate of a $3 \times 3$ matrix is given by:

\[
\text{adj}(A) = \begin{bmatrix}
A_{11} & A_{21} & A_{31} \\
A_{12} & A_{22} & A_{32} \\
A_{13} & A_{23} & A_{33}
\end{bmatrix}
\]

where $A_{ij}$ is the cofactor of element $a_{ij}$ in matrix $A$.

Note that not all matrices have an inverse. A matrix is invertible if and only if its determinant is non-zero.

\subsubsection*{Attributes}

\begin{itemize}
  \item It is unique.
  \item The inverse of $A^{-1}$ is $A$ itself.
\end{itemize}

\hrulefill

\textbf{Claim.} Suppose A is invertible. Then its inverse is unique.

\textbf{Proof.} Suppose $B$ and $C$ are both inverses of $A$.
Then $B = BI = B(AC) = (BA)C = IC = C$.

\Remark{left inverse = right inverse = inverse}

\textbf{Claim.} The inverse of $A^{-1}$ is $A$ itself.

\textbf{Proof.} $AA^{-1} = I$ and $A^{-1}A = I$.

\textbf{Claim.} If $A$ is invertible, then the one and only solution to $A\underline{x} = \underline{b}$ is $\underline{x} = A^{-1}\underline{b}$.

\textbf{Proof.} $A\underline{x} = \underline{b} \Rightarrow A^{-1}A\underline{x} = A^{-1}\underline{b} \Rightarrow \underline{x} = A^{-1}\underline{b}$.

\textbf{Claim.} Suppose there is a nonzero solution $\underline{x}$ to $A\underline{x} = \underline{0}$ (homogeneous equation).
Then $A$ is not invertible.

\textbf{Proof.} If $A$ is invertible, then $A^{-1}$ exists.
Then $A^{-1}A\underline{x} = A^{-1}\underline{0} \Rightarrow \underline{x} = \underline{0}$.

\textbf{Claim.} A diagonal matrix has an inverse provided no diagonal entries are zero.

\textbf{Proof.} \\
If

\[
A = \begin{bmatrix}
  d_1 & 0 & \cdots & 0 \\
  0 & d_2 & \cdots & 0 \\
  \vdots & \vdots & \ddots & \vdots \\
  0 & 0 & \cdots & d_n
\end{bmatrix}
\]

then

\[
A^{-1} = \begin{bmatrix}
  \frac{1}{d_1} & 0 & \cdots & 0 \\
  0 & \frac{1}{d_2} & \cdots & 0 \\
  \vdots & \vdots & \ddots & \vdots \\
  0 & 0 & \cdots & \frac{1}{d_n}
\end{bmatrix}
\]

\textbf{Claim.} If $A$ and $B$ are invertible, then $AB$ is invertible and $(AB)^{-1} = B^{-1}A^{-1}$.

\textbf{Proof.} 

$(B^{-1}A^{-1})(AB) = B^{-1}(A^{-1}A)B = B^{-1}IB = I$ \\
$(AB)(B^{-1}A^{-1}) = A(BB^{-1})A^{-1} = AIA^{-1} = I$

\Remark{$(ABC)^{-1} = C^{-1}B^{-1}A^{-1}$}

\subsection*{Gauss-Jordan Elimination}

Given $A$, we want to find its inverse $A^{-1}$. $AA^{-1} = I$

\[
A
\begin{bmatrix}
  \underline{col_1} & \underline{col_3} & \underline{col_3} \\
\end{bmatrix} = 
\begin{bmatrix}
  1 & 0 & 0 \\
  0 & 1 & 0 \\
  0 & 0 & 1
\end{bmatrix} = 
\begin{bmatrix}
  \underline{e_1} & \underline{e_3} & \underline{e_3} \\
\end{bmatrix}
\]

$
A = 
\begin{bmatrix}
  2 & -1 & 0 \\ 
  -1 & 2 & -1 \\
  0 & -1 & 2
\end{bmatrix}
$

\[
\begin{bmatrix}
  \boxed{2} & -1 & 0 & \vdots & 1 & 0 & 0 \\
  -1 & 2 & -1 & \vdots & 0 & 1 & 0 \\
  0 & -1 & 2 & \vdots & 0 & 0 & 1
\end{bmatrix}
\]
$\Rightarrow$
\[
\begin{bmatrix}
  2 & -1 & 0 & \vdots & 1 & 0 & 0 \\
  0 & \boxed{\frac{3}{2}} & -1 & \vdots & \frac{1}{2} & 1 & 0 \\
  0 & -1 & 2 & \vdots & 0 & 0 & 1
\end{bmatrix}
\]
$\Rightarrow$
\[
\begin{bmatrix}
  2 & -1 & 0 & \vdots & 1 & 0 & 0 \\
  0 & \frac{3}{2} & -1 & \vdots & \frac{1}{2} & 1 & 0 \\
  0 & 0 & \boxed{\frac{4}{3}} & \vdots & \frac{1}{3} & \frac{2}{3} & 1
\end{bmatrix}
\]
Until here is Gauss \\
$\Rightarrow$
\[
\begin{bmatrix}
  2 & -1 & 0 & \vdots & 1 & 0 & 0 \\
  0 & \frac{3}{2} & 0 & \vdots & \frac{3}{4} & \frac{3}{2} & \frac{3}{4} \\
  0 & 0 & \boxed{\frac{4}{3}} & \vdots & \frac{1}{3} & \frac{2}{3} & 1
\end{bmatrix}
\]
$\Rightarrow$
\[
\begin{bmatrix}
  2 & 0 & 0 & \vdots & \frac{3}{2} & 1 & \frac{1}{2} \\
  0 & \boxed{\frac{3}{2}} & 0 & \vdots & \frac{3}{4} & \frac{3}{2} & \frac{3}{4} \\
  0 & 0 & \boxed{\frac{4}{3}} & \vdots & \frac{1}{3} & \frac{2}{3} & 1
\end{bmatrix}
\]
$\Rightarrow$
\[
\begin{bmatrix}
  1 & 0 & 0 & \vdots & \frac{3}{4} & \frac{1}{2} & \frac{1}{4} \\
  0 & 1 & 0 & \vdots & \frac{1}{2} & 1 & \frac{1}{2} \\
  0 & 0 & 1 & \vdots & \frac{1}{4} & \frac{1}{2} & \frac{3}{4}
\end{bmatrix}
\]
Until here is Jordan \\

$
col_1 = 
\begin{bmatrix}
  \frac{3}{4} \\
  \frac{1}{2} \\
  \frac{1}{4}
\end{bmatrix}
$, 
$
col_2 =
\begin{bmatrix}
  \frac{1}{2} \\
  1 \\
  \frac{1}{2}
\end{bmatrix}
$,
$
col_3 =
\begin{bmatrix}
  \frac{1}{4} \\
  \frac{1}{2} \\
  \frac{3}{4}
\end{bmatrix}
$,
$
A^{-1} =
\begin{bmatrix}
  \frac{3}{4} & \frac{1}{2} & \frac{1}{4} \\
  \frac{1}{2} & 1 & \frac{1}{2} \\
  \frac{1}{4} & \frac{1}{2} & \frac{3}{4}
\end{bmatrix}
$

\textbf{Claim.} A matrix is invertible if and only if (iff.) it is nonsingular.

\subsection*{Elimination = Factorization: $A = LU$}

TODO

\textbf{Claim.} If $A = L_1D_1U_1$ and $A = L_2D_2U_2$, where the $L$'s are lower-triangular with unit diagonal,
the $U$'s are upper-triangular with unit diagonal, and the $D$'s are diagonal matrices with no zeros on the diagonal,
then $L_1 = L_2$, $D_1 = D_2$, and $U_1 = U_2$.


\end{document}