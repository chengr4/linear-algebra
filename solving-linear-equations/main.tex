\documentclass[12pt,a4paper]{article}
\usepackage{amsfonts, amssymb, amsmath}
\usepackage{fullpage}
\usepackage{parskip} % skip a line instead of indenting
\usepackage{amsthm}
\usepackage{xcolor}

\newtheorem*{rem}{Remark}

\title{Solving Linear Equations}
\author{R4 Cheng}
\date{\today}

\newcommand{\remark}[1]{
  \begin{rem}
    \color{cyan}
    #1
  \end{rem}
}

\begin{document}
\maketitle

\subsection*{Matrix Operations}

$
\begin{bmatrix}
  1 & 2 & -4 \\
  -2 & 3 & 1 \\
  4 & 1 & 2
\end{bmatrix}
\begin{bmatrix}
  x_1 & x_2 & x_3 \\
  y_1 & y_2 & y_3 \\
  z_1 & z_2 & z_3 
\end{bmatrix} = 
\begin{bmatrix}
  & \\
  \underline{c_1} & \underline{c_2} & \underline{c_3} \\
  &
\end{bmatrix}
$

$
\underline{c_2} = 
x_2
\begin{bmatrix}
  1 \\
  2 \\
  4
\end{bmatrix} + 
y_2
\begin{bmatrix}
  2 \\
  3 \\
  1
\end{bmatrix} +
z_2
\begin{bmatrix}
  -4 \\
  1 \\
  2
\end{bmatrix}
$

$
\begin{bmatrix}
  a & b & c \\
\end{bmatrix}
\begin{bmatrix}
  1 & 2 & -4 \\
  -2 & 3 & 1 \\
  4 & 1 & 2
\end{bmatrix} = 
a
\begin{bmatrix}
  1 & 2 & -4 \\
\end{bmatrix} +
b
\begin{bmatrix}
  -2 & 3 & 1 \\
\end{bmatrix} +
c
\begin{bmatrix}
  4 & 1 & 2
\end{bmatrix}
$


\subsection*{Properties of Matrices}

$A(BC) = (AB)C$ (Associative law holds)

$AB \neq BA$ (Commutative law does not hold)

$C(A+B) = CA + CB$ or $(A+B)C = AC + BC$ (Distributive laws hold)







\end{document}