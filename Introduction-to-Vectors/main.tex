\documentclass[12pt,a4paper]{article}
\usepackage{amsfonts, amssymb, amsmath}
\usepackage{fullpage}
\usepackage{parskip} % skip a line instead of indenting
\usepackage{amsthm}
\usepackage{xcolor}

\newtheorem*{rem}{Remark}

\title{Introduction to Vectors}
\author{R4 Cheng}
\date{\today}

\newcommand{\remark}[1]{
  \begin{rem}
    \color{cyan}
    #1
  \end{rem}
}

\begin{document}
\maketitle

$ A \underline{x} = \underline{b} $

\begin{itemize}
  \item $A$ is a matrix
  \item $\underline{x}$ is a vector
  \item $\underline{b}$ is an another vector
\end{itemize}

\remark{If the three column vectors do not lie in the same plain, we can solve $A\underline{x}=\underline{b}$ for every $\underline{b}$}

\subsection*{Guassian Elimination}

\begin{align}
  2x + 4y - 2z &= 2 \quad \text{2 is pivot} \\
  4x + 9y - 3z &= 8 \quad \text{$-2 \times (1)$} \\
  -2x - 3y + 7z &= 10 \quad \text{$+ (1)$}
\end{align}
$\Rightarrow$
\begin{align}
  2x + 4y - 2z &= 2 \\
  y + z &= 4 \\
  y + 5z &= 12
\end{align}
$\Rightarrow$
\begin{align}
  2x + 4y - 2z &= 2 \\
  y + z &= 4 \\
  4z &= 8
\end{align}

pivots: 2, 1, 4 (They are meaningful) \\
Do backward substitution to get: $z = 2, y = 2, x = -1$

\[
\begin{bmatrix}
  \boxed{2} & 4 & -2 & \vdots & 2 \\
  4 & 9 & -3 & \vdots & 8 \\
  -2 & -3 & 7 & \vdots & 10
\end{bmatrix}
\]
$\Rightarrow$
\[
\begin{bmatrix}
  2 & 4 & -2 & \vdots & 2 \\
  0 & \boxed{1} & 1 & \vdots & 4 \\
  0 & 1 & 5 & \vdots & 12
\end{bmatrix}
\]
$\Rightarrow$
\[
\begin{bmatrix}
  2 & 4 & -2 & \vdots & 2 \\
  0 & 1 & 1 & \vdots & 4 \\
  0 & 0 & \boxed{4} & \vdots & 8
\end{bmatrix}
\]

In general,

$[A \vdots b] \Rightarrow [\bigsqcup \vdots \underline{c}]$

\remark{$\bigsqcup$ means upper-triangualr matrix}
\remark{In order to solve the unknowns, pivots cannot be 0}



\end{document}