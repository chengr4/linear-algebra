\documentclass[12pt,a4paper]{article}
\usepackage{amsfonts, amssymb, amsmath}
\usepackage{fullpage}
\usepackage{parskip} % skip a line instead of indenting
\usepackage{amsthm}
\usepackage{xcolor}
\usepackage{tikz}

\newtheorem*{rem}{Remark}

\title{Vector Spaces and Subspaces}
\author{R4 Cheng}
\date{\today}

\newcommand{\Remark}[1]{
  \begin{rem}
    \color{cyan}
    #1
  \end{rem}
}

\begin{document}
\maketitle

$\mathbb{R}^n = $ all (column) vectors with n (real) components. \\
$= \{ (v_1, v_2, \cdots, v_n): v_i \in \mathbb{R}, i = 1, 2, \cdots, n \}$

\[
\begin{bmatrix}
  4 \\
  \pi \\
\end{bmatrix} \in \mathbb{R}^2,
\quad
(1, 1, 0, 1, 1) \in \mathbb{R}^5
\]
\[
\begin{tikzpicture}
  % Draw x-axis
  \draw[->] (-2, 0) -- (2, 0) node[right] {$x$};
  % Draw y-axis
  \draw[->] (0, -2) -- (0, 2) node[above] {$y$};
  
  % Draw vector v
  \draw[->, red] (0, 0) -- (1, 1) node[midway, below right] {$\mathbf{v}$};
  \draw[->, pink] (0, 0) -- (3, 3) node[midway, above right] {$\mathbf{3v}$};
  
  % Draw vector w
  \draw[->, blue] (0, 0) -- (1, 2) node[midway, above left] {$\mathbf{w}$};
  
  % Draw vector v + vector w
  \draw[->, purple] (0, 0) -- (2, 3) node[above right] {$\mathbf{v} + \mathbf{w}$};
\end{tikzpicture}
\]

\subsubsection*{Vector Space $V$}

$V$: a set of vectors
\begin{enumerate}
  \item Two operations:
  \begin{itemize}
    \item vector addition: $\underline{v}, \underline{w} \in V \Rightarrow \underline{v} + \underline{w} \in V$
    \item scalar multiplication: $c \underline{v} \in V$
  \end{itemize}
  \item Eight rules:
  \begin{enumerate}
    \item $\underline{v} + \underline{w} = \underline{w} + \underline{v}$ (commutative)
    \item $(\underline{v} + \underline{w}) + \underline{z} = \underline{v} + (\underline{w} + \underline{z})$ (associative)
    \item There is a unique "zero vector" \underline{0} such that $\underline{v} + \underline{0} = \underline{v}$ for all $\underline{v} \in V$
    \item For each $\underline{v}$, there is a unique vector $-\underline{v}$ such that $\underline{v} + (-\underline{v}) = \underline{0}$
    \item $1 \times \underline{v} = \underline{v}$
    \item $(c_1c_2)\underline{v} = c_1(c_2\underline{v})$
    \item $c(\underline{v} + \underline{w}) = c\underline{v} + c\underline{w}$
    \item $(c_1 + c_2)\underline{v} = c_1\underline{v} + c_2\underline{v}$
  \end{enumerate}
\end{enumerate}

$\Rightarrow 0 \times \underline{v}  = \underline{0}$  (not 0) \\
$\Rightarrow (-1)\underline{v} = -\underline{v}$

Examples:

\begin{itemize}
  \item $\mathbb{R}^n$ is a vector space
  \item $M =$ \{all real $2 \times 2$ matrices\} is a vector space
  \item $F =$ \{all real functions $f(x)$ \} is a vector space
  \item $z = \{ \underline{0} \}$ is a vector space
\end{itemize}

\subsubsection*{Subspaces}

\textbf{Def.} A subset W of a vector space V is a \underline{subspaces} if W itself is a vector space.

\textbf{Claim} Every subspace contains the zero vector.

\textbf{Proof} $ 0 \times \underline{v} = \underline{0} \in W$

\end{document}